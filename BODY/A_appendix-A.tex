\chapter{Spinors in various dimensions}
\label{ap:spinors}

In this appendix we will juntify some claims about spinors in general even dimensions that are used throughout the thesis.

It is well known that the Dirac representation is not irreducible in even dimensions, in which a chirality projection exists into the two different Weyl basis. There is also always a Majorana condition that can induces a real structure in the spinor spaces of all dimensions, but is only truncates the degrees of freedom of Weyl spinors in some dimensions. All these topics will be covered in the following sections in detail with the main objective of describing irreducible spinors in all even dimensions $D \leq 10$.

% START APPENDIX
\section{Weyl Spinors in $D = 2,4,6,8,10$}

We start with the Clifford algebra,

\begin{equation}
    \label{eq:cliff_alg}
    \{ \Gamma^\mu, \Gamma^\nu \} = 2\eta^{\mu \nu},
\end{equation}

and let the metric be $\eta^{\mu \nu} = \text{diag}\left( -1,1,\dots,1 \right)$ the $D = 2k + 2$ dimensional Minkowski metric.

We first make a change of basis for the algebra elements so that the fundamental representation can be directly extracted from the algebra. Take the following linear combinations,

\begin{align}
    \label{eq:cliff_rebase}
    & \Gamma^{0 \pm}=\frac{1}{2}\left( \pm \Gamma^0+\Gamma^1\right) \\
    & \Gamma^{a \pm}=\frac{1}{2}\left(\Gamma^{2 a} \pm i \Gamma^{2 a+1}\right), \quad a=1, \ldots, k
\end{align}

Now the Clifford algebra \ref{eq:cliff_alg} can be stated in terms of the new operators as,

\begin{equation}
    \begin{aligned}
    & \left\{\Gamma^{a+}, \Gamma^{b-}\right\}=\delta^{a b} \\
    & \left\{\Gamma^{a+}, \Gamma^{b+}\right\}=\left\{\Gamma^{a-}, \Gamma^{b-}\right\}=0 .
    \end{aligned}
\end{equation}

We can see that these operators are raising and lowering operators for $k+1$ different eigenvalues. We can write an arbitrary basis element of this representation as $\ket{s_0, \dots, s_{k}}$, with $s_i = \pm 1$. An arbitrary spinor in this representation is then in the complex span of this basis, which has $2^{k+1}$ complex components. This is what is called a Dirac spinor, or $\mathbf{2^{k+1}}_{\text{Dirac}}$.

These eigenvalues $s_i$ are actually eigenvalues of rotations in planes given by the grouping of \ref{eq:cliff_rebase}. To see this explicitly, we need to recover the Lorentz algebra from the Clifford algebra. Define,

\begin{equation}
    \Sigma^{\mu v}=-\frac{i}{4}\left[\Gamma^\mu, \Gamma^v\right]
\end{equation}

The elements $\Sigma^{\mu v}$ define the Lorentz algebra os $SO(1, 2k+1)$. The generators $\Sigma^{2a,2a+1}$ commute and have eigenvalues proportional to $s_a$ when acting on the basis element $\ket{s_0, \dots, s_{k}}$. To be precise, the operator,

\begin{equation}
    S_a \equiv i^{\delta_{a, 0}} \Sigma^{2 a, 2 a+1}=\Gamma^{a+} \Gamma^{a-}-\frac{1}{2}
\end{equation}

has eigenvalue $s_a$.

Next, we can define the chirality operator,

\begin{equation}
    \label{eq:chirality_op}
    \Gamma=i^{-k} \Gamma^0 \Gamma^1 \ldots \Gamma^{d-1}
\end{equation}

which has the properties,

\begin{equation}
    \label{eq:chiral_properties}
    (\Gamma)^2=1, \quad\left\{\Gamma, \Gamma^\mu\right\}=0, \quad\left[\Gamma, \Sigma^{\mu v}\right]=0
\end{equation}

From the first property, we see that $\Gamma$ has eigenvalues $\pm 1$. From the rest we see that we can split the basis $\ket{s_0,\dots,s_k}$ into two subspaces according to the eigenvalues of $\Gamma$. We can rewrite the chirality operator in terms of the rotation generators $S_a$ as follows,

\begin{equation}
    \Gamma=2^{k+1} S_0 S_1 \ldots S_k
\end{equation}

which allows us to identify the two chiralities as $+1$ when the product of the $s_a$ is positive and $-1$ when it is negative. The two subspaces defined by the chirality operator are called Weyl representations and are labeled as $\mathbf{2^k}_s$ and $\mathbf{2^k}_c$ for the $+1$ and $-1$ subspaces respectively. So we can finally state that the dirac representation splits into Weyl representations as,

\begin{equation}
    \mathbf{2^{k+1}}_\text{Dirac} = \mathbf{2^k}_s \oplus \mathbf{2^k}_c
\end{equation}

The staple example in string theory is $D=10$, where we have the decomposition,

\begin{equation}
    \mathbf{32}_{\text{Dirac}} = \mathbf{16}_s \oplus \mathbf{16}_c
\end{equation}

\section{Majorana condition}

Up until now, we have been able to define Weyl spinors, that have a well defined chirality and have half the degrees of freedom of Dirac spinors. It will be shown in this section that a real structure can be imposed for some values of $D$, effectively halving the degrees of freedom of a Dirac spinor.

From the definition \ref{eq:cliff_rebase}, and the action on the basis elements $\ket{s_0,\dots, s_k}$, we see that as a matrix, $\Gamma^{a\pm}$ are real. For the original gamma matrices defined in \ref{eq:cliff_alg}, we see that the matrices $\Gamma^{2a}$ are real, while the matrices $\Gamma^{2a+1}$ are purely imaginary.

Now, take all of the purely imaginary matrices, and define the operators,

\begin{equation}
    B_1=\Gamma^3 \Gamma^5 \ldots \Gamma^{d-1}, \quad B_2=\Gamma B_1
\end{equation}

From the commutation relations of the Gamma matrices we have that,

\begin{equation}
    B_1\Gamma^\mu B_1^{-1}=(-1)^k\Gamma^{\mu*},\quad B_2\Gamma^\mu B_2^{-1}=(-1)^{k+1}\Gamma^{\mu*},
\end{equation}

and for the Lorentz generators,

\begin{equation}
    \label{eq:lorentz_conjugate}
    B\Sigma^{\mu\nu}B^{-1}=-\Sigma^{\mu\nu*},
\end{equation}

for either $B_1$ or $B_2$.

Now, take a Dirac spinor $\xi$, and make a change of basis given by $\xi \rightarrow B\xi$. Then, by the trnasformation rules of $\xi$ and the relation \ref{eq:lorentz_conjugate}, we see that $B\xi$ trnasforms as a conjugate spinor $\xi^*$.

The fact that we are able to linearly transform into conjugate spinors means that we may be able to relate the real and imaginary components of a spinor in a way consistent with Lorentz transformations. Concretely, we propose the Majorana condition,

\begin{equation}
    \xi^* = B\xi.
\end{equation}

Now, taking the conjugates $(B\xi)^* = \xi = B^*B\xi$, so $B^* B = 1$. Now, we can calculate explicitly using the definitions for $B_1$ and $B_2$ that,

\begin{equation}
    B_1^*B_1=(-1)^{k(k+1)/2},\quad B_2^*B_2=(-1)^{k(k-1)/2}.
\end{equation}

So the condition $B^* B = 1$ can only be satisfied with $k = 0, 1, 3$ (mod 4). Which means that Majorana spinors can exist in $D = 2, 4, 8$ (mod 8) but not in $D = 6$ (mod 8).

We are ultimately interested in Majorana-Wey spinors, so we need to discuss whether a Majorana condition can be applied to a Weyl spinor. Take the chirality matrix $\Gamma$. We need to check if a Majorana change of basis can keep the chirality consistant. Otherwise, there would be mixing between right and left moving spinors, making them unconsistant with Lorentz trnasofrmations.

From the properties of the chirality matrix \ref{eq:chiral_properties}, we calculate,

\begin{equation}
    B_1\Gamma B_1^{-1}=B_2\Gamma B_2^{-1}=(-1)^k\Gamma^*,
\end{equation}

so when $k$ is even, each Weyl representation transforms as its own conjugate, while for $k$ odd, the transformation rules get exchanged.

This automatically forbids Majorana-Weyl spinors from existing in $D = 4, 6$ (mod 8), so that both conditions are only compatible in $D = 2$ (mod 8). This is a very interesting result, because superstring theory is formulated from MW spinors in the worldsheet, this is $D = 2$, and produces MW spinors in spacetime, which is necessarily $D = 10$, it is quite a beautiful coincidence.


\section{Table of irreducible spinors in even dimensions}

To wrap up this brief lesson on spinors, we can list all of the minimal degrees of freedom of a spinor in any (even) dimension mod 8. For the sake of completeness I will also list the minimal representations in odd dimensions without any derivation (see \cite{Polchinski_1998} for details),

\begin{table}[h]
    \centering
    \begin{tabular}{ccccc}
    $d$ & Majorana & Weyl    & Majorana-Weyl & minimal dof \\ \hline
    2   & yes      & self    & yes           & 1           \\
    3   & yes      & -       & -             & 2           \\
    4   & yes      & complex & -             & 4           \\
    5   & -        &         & -             & 8           \\
    6   & -        & self    & -             & 8           \\
    7   & -        & -       & -             & 16          \\
    8   & yes      & complex & -             & 16          \\
    9   & yes      & -       & -             & 16          \\
    10  & yes      & self    & yes           & 16         
\end{tabular}
\end{table}