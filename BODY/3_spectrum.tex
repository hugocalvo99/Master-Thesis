%%%%%%%%%%%%%%%%%%%%%%%%%%%%%%%%%%%%%%%%%%%%%%%%%%
%%%%		~~~~ Method ~~~~
%%%%%%%%%%%%%%%%%%%%%%%%%%%%%%%%%%%%%%%%%%%%%%%%%%


\chapter{Open string spectrum}
\label{chap:spectrum}
\pagestyle{fancy}

In this chapter we will describe the spectrum of D-brane systems in the context of type IIB String Theory. We start by discussing single brane spectrums, and then move on to general brane configurations, to conclude with the main example of this thesis, the D1/D5 brane system.

\section{Dp-Dp spectrum}

%Discuss the spectrum of a single Dp-brane. Start with D9 SYM and dimensionally reduce.

Consider a single D9-brane. This equates to considering Neumann boundary conditions in all directions of a string. The NS vacuum is a scalar $\ket{0}$, while the R sector is a spinor $\ket{a}$, given by $SO(2)$ eigenvalues $s_0, s_1, s_2, s_3, s_4$. The massless spectrum can be found from the mass formula $\alpha ' M^2 = N + 1/2$. This leads to the modes $b_{-1/2}^\mu \ket{0}$ and $\ket{a}$.

The representation theory of the massless spectrum turns out to be straight forward. There is a vector and a fermion in D = 10, composed by $b_{-1/2}^\mu \ket{0}$ and $\ket{a}$ respectively.

To obtain the spectrum of an arbitrary Dp-brane we can perform dimensional reduction over the D9 spectrum we already constructed. By dimensional reduction we mean splitting the D9 symmetry group $SO(1,9)$ into the transverse and world-volume symmetry groups of the lower dimensional Dp-brane, namely, $SO(1,p) \times SO(9-p)$. 

Starting with a vector in the $\mathbf{10}$ of $SO(1,9)$, it decomposes into a $(\mathbf{p},\mathbf{1}) \oplus (\mathbf{1}, \mathbf{9-p})$ of $SO(1,p)\times SO(9-p)$. The original R vacua was a Majorana-Weyl spinor of $SO(1,9)$, and depending on $p$ it will decompose into the corresponding irreducible spinors of $SO(1,p)\times SO(9-p)$. The dimensional reduction of spinors is discussed in detail in \ref{ap:spinors}.

This is already the full spectrum of a single Dp-brane. Extending it to a stack of $Q_p$ Dp-branes adds a Chan-Paton factor to both string ends, which labels the adjoint representation of $U(Q_p)$. 

We can count on-shell degrees of freedom to make sure the spectrum is supersymmetric. For a explicit example, we can consider a single D5-brane. By on-shell we refer to adopting light-cone gauge, meaning that $SO(1,5) \rightarrow SO(4)$, effectively identifying 2 degrees of freedom for the vector, and specifying a $s_0$ eigenvalue for the spinor.

\begin{align*}
    (\mathbf{6},\mathbf{1}) \rightarrow (\mathbf{4},\mathbf{1}\\
    (\mathbf{1},\mathbf{4}) \rightarrow (\mathbf{1}, \mathbf{4})\\
    (\mathbf{4}_s,\mathbf{2}_s) \rightarrow (\mathbf{2}_s,\mathbf{2}_s) \\
    (\mathbf{4}_c,\mathbf{2}_c) \rightarrow (\mathbf{2}_c,\mathbf{2}_c) \\
\end{align*}

Now if we count degrees of freedom in the right side, we find that there are 8 matching bosonic and fermionic degrees of freedom, meaning that there can be 8 on-shell supercharges in the theory. A general discussion can be made for any $p$ to find the same 8 possible on-shell supercharges. 

\section{Dp-D(p+4) spectrum}

%Start by finding the vacuum. See how it is like a full R vacuum split as a tensor product

\section{D1/D5 spectrum}